\documentclass[a4paper,70pt]{article}

%%% Работа с русским языком
\usepackage{cmap}					% поиск в PDF
\usepackage{mathtext} 				% русские буквы в фомулах
\usepackage[T2A]{fontenc}			% кодировка
\usepackage[utf8]{inputenc}			% кодировка исходного текста
\usepackage[english,russian]{babel}	% локализация и переносы
%% Шрифты
\usepackage{euscript}	 % Шрифт Евклид
\usepackage{mathrsfs} % Красивый матшрифт
%%% Работа с картинками
\usepackage{graphicx}  % Для вставки рисунков
\graphicspath{{images/}{images2/}}  % папки с картинками
\setlength\fboxsep{3pt} % Отступ рамки \fbox{} от рисунка
\setlength\fboxrule{1pt} % Толщина линий рамки \fbox{}
\usepackage{wrapfig} % Обтекание рисунков и таблиц текстом
\usepackage{gensymb}

\title{Задания}
\date{}


\begin{document}
\fontsize{18}{16pt}\selectfont
\maketitle
1. $448,7405:2,21=$ 
\paragraph{}
Свойства сложения и умножения:
\paragraph{}
a+b=b+a; ab=ba - переместительное
\paragraph{}
a+(b+с)=(a+b)+c; a(bc)=(ab)c - сочетательное
\paragraph{}
a(b+с)=ab+ac - распределительное (раскрытие скобок/вынесение за скобки)
\paragraph{}

2. $9\frac{3}{8}*2\frac{5}{7}-2\frac{5}{7}*7=$
\paragraph{}
3. $7,8*6,3+7,8*13,7=$
\paragraph{}
4. Группа школьников отправляется от пристани на моторной лодке по течению реки с условием вернуться через 4 часов. Скорость течения реки 3 км/ч, скорость лодки в стоячей воде 12 км/ч. На какое наибольшее расстояние школьники могут отплыть от пристани? Подсказка: за x нужно взять расстояние в одну сторону.

\end{document}

